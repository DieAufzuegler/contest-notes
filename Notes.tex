\documentclass[11pt]{article}

\usepackage{amsmath}
\usepackage[a4paper,includeheadfoot,margin=1cm]{geometry}
\usepackage{graphicx}
\usepackage[utf8]{inputenc}
\usepackage{listings}
\usepackage{xcolor}

\setlength{\columnsep}{1cm}

\definecolor{mygreen}{rgb}{0,0.6,0}
\definecolor{mygray}{rgb}{0.5,0.5,0.5}
\definecolor{mymauve}{rgb}{0.58,0,0.82}

\lstset{ %
  backgroundcolor=\color{white},   % choose the background color; you must add \usepackage{color} or \usepackage{xcolor}; should come as last argument
  basicstyle=\footnotesize\ttfamily,	   % the size of the fonts that are used for the code
  breakatwhitespace=true,         % sets if automatic breaks should only happen at whitespace
  breaklines=true,                 % sets automatic line breaking
  captionpos=b,                    % sets the caption-position to bottom
  commentstyle=\color{mygreen},    % comment style
  deletekeywords={...},            % if you want to delete keywords from the given language
  escapeinside={\%*}{*)},          % if you want to add LaTeX within your code
  extendedchars=true,              % lets you use non-ASCII characters; for 8-bits encodings only, does not work with UTF-8
  frame=single,                    % adds a frame around the code
  keepspaces=true,                 % keeps spaces in text, useful for keeping indentation of code (possibly needs columns=flexible)
  keywordstyle=\color{blue},	   % keyword style
  language=Octave,                 % the language of the code
  morekeywords={*,...},            % if you want to add more keywords to the set
  numbers=left,                    % where to put the line-numbers; possible values are (none, left, right)
  numbersep=5pt,                   % how far the line-numbers are from the code
  numberstyle=\tiny\color{mygray}, % the style that is used for the line-numbers
  rulecolor=\color{black},         % if not set, the frame-color may be changed on line-breaks within not-black text (e.g. comments (green here))
  showspaces=false,                % show spaces everywhere adding particular underscores; it overrides 'showstringspaces'
  showstringspaces=false,          % underline spaces within strings only
  showtabs=false,                  % show tabs within strings adding particular underscores
  stepnumber=2,                    % the step between two line-numbers. If it's 1, each line will be numbered
  stringstyle=\color{mymauve},     % string literal style
  tabsize=2,                       % sets default tabsize to 2 spaces
  %title=\lstname                   % show the filename of files included with \lstinputlisting; also try caption instead of title
}


%opening
\title{Contest Notes\\Raiders of the Lost Array\\Frankfurt University of Applied Sciences}
\author{Marcus Legendre, Tobias Lehnert, Janik Hammerschmitt}
\date{}

\begin{document}

\maketitle

\tableofcontents


\section{Input/Output}

\subsection{Read from stdin until EOF in Python}
\lstinputlisting[language=Python]{code-snippets/read-stdin-until-eof.py}

\subsection{C++ iomanip}
\lstinputlisting[language=C++]{code-snippets/iomanip.cpp}

\section{Number Theory}

\subsection{Greatest Common Divisor and Euler's Totient Function}
\lstinputlisting[language=C++]{code-snippets/eulerstotient.cpp}

\subsection{Factorisation}
\lstinputlisting[language=C++]{code-snippets/factorisation.cpp}

\subsection{Prime Number Tester}
\lstinputlisting[language=C++]{code-snippets/primenumbertester.cpp}


\section{Dynamic Programming}

\subsection{Catalan Numbers $\mathcal{O}(n^2)$}
\lstinputlisting[language=C++]{code-snippets/catalan-numbers.cpp}

\subsection{Catalan Numbers $\mathcal{O}(n)$}
\lstinputlisting[language=Python]{code-snippets/catalan-numbers.py}

\subsection{Knapsack}
\lstinputlisting[language=Python]{code-snippets/knapsack.py}

\subsection{Longest Common Subsequence}
\lstinputlisting[language=Python]{code-snippets/longest-common-subsequence.py}


\section{Geometry}

\subsection{Convex Hull}
\lstinputlisting[language=C++]{code-snippets/convex-hull.cpp}

\subsection{Polygon Area}
\lstinputlisting[language=C++]{code-snippets/polygon-area.cpp}


\section{Graph Theory}

\subsection{Breadth First Search}
\lstinputlisting[language=C++]{code-snippets/breadth-first-search.cpp}

\subsection{Depth First Search}
\lstinputlisting[language=C++]{code-snippets/depth-first-search.cpp}

\subsection{Dijkstra}
\lstinputlisting[language=Python]{code-snippets/dijkstra.py}


\section{Formulas}

\subsection{Combinatorics}
\begin{tabular}{ | l | c | c |}
  \hline
   & Mit Reihenfolge & Ohne Reihenfolge \\ \hline
   Mit Wdh & $n^k$ & $\left( {n+k-1} \atop k \right)$  \\ \hline
   Ohne Wdh & $\frac{n!}{\left( n-k \right) !}$ & $\frac{n!}{\left( n-k \right) !\cdot k!} = \left( n \atop k \right)$  \\ \hline
\end{tabular}


\subsection{Geometry}
$$\text{Radian} = \frac{\text{Angle}}{180}\cdot \pi$$

\end{document}
